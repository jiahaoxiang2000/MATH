\chapter{Vector Spaces and linear mappings}
\label{chap:VectorSpaces}

Broadly speaking, a vector space over a field $F$ refers to a set $V$ together with two operations:
\begin{itemize}
  \item Vector addition $+: V \times V \rightarrow V$, denoted $(v_1, v_2) \mapsto v_1 + v_2$, satisfying associativity, commutativity, and the existence of inverses;
  \item Scalar multiplication $\cdot: F \times V \rightarrow V$, denoted $(t, v) \mapsto t \cdot v = tv$, satisfying associativity and distributivity over addition.
\end{itemize}

If a mapping $T: V \rightarrow W$ between vector spaces satisfies the identities
\begin{align*}
  T(v_1 + v_2) &= T(v_1) + T(v_2),\\
  T(tv) &= tT(v),
\end{align*}
then $T$ is called a linear mapping.

\section{Introduction: Back to the system of linear equations}

\begin{align}
  a_{11}X_1 + \cdots + a_{1n}X_n &= b_1 \nonumber\\
  a_{21}X_1 + \cdots + a_{2n}X_n &= b_2 \nonumber \\
  &\vdots\nonumber \\
  a_{m1}X_1 + \cdots + a_{mn}X_n &= b_m \label{eq:linear-system-2}
\end{align}

\begin{definition}
  Consider a system of $n$ linear equations over a field $F$ in the form (\ref{eq:linear-system-2}). If $b_1 = \cdots = b_m = 0$, then the system is called homogeneous.
\end{definition}

Given $n,m \in \mathbb{Z}_{\geq 1}$ and a family of coefficients $(a_{ij})_{1 \leq i \leq m, 1 \leq j \leq n}$ where $a_{ij} \in F$, define the mapping
\begin{align*}
  T: F^n &\rightarrow F^m\\
  (x_j)_{j=1}^n &\mapsto \left(\sum_{j=1}^n a_{1j}x_j, \ldots, \sum_{j=1}^n a_{mj}x_j\right).
\end{align*}

\begin{definition}
  Let $T: F^n \rightarrow F^m$ correspond to a homogeneous system of linear equations as described above. If $v_1, \ldots, v_h \in F^n$ are all solutions of the system, and every solution $x \in F^n$ can be uniquely expressed through addition and scalar multiplication as
  \begin{align}
    x = \sum_{i=1}^h t_i v_i, \quad t_1, \ldots, t_h \in F,
  \end{align}
  where the tuple $(t_1, \ldots, t_h)$ is uniquely determined by $x$, then $v_1, \ldots, v_h$ is called a fundamental system of solutions for the homogeneous system.
\end{definition}

\begin{proposition}
  Consider a homogeneous system of $n$ linear equations in the form (Eq.\ref{eq:linear-system-2}), where $\bf{b} = 0$. If the reduced row echelon matrix obtained by elimination has $r$ pivot elements, then the corresponding homogeneous system has a fundamental system of solutions $v_1, \ldots, v_{n-r}$.
\end{proposition}

\section{Vector spaces}

\begin{definition}
  A vector space over a field $F$, also called an $F$-vector space, is a tuple $(V, +, \cdot, 0_V)$ where $V$ is a set, $0_V \in V$, and operations $+: V \times V \rightarrow V$ and $\cdot: F \times V \rightarrow V$ are written as $(u,v) \mapsto u + v$ and $(t,v) \mapsto t \cdot v$ respectively, satisfying the following conditions:

  1. Addition satisfies:
  \begin{itemize}
    \item Associativity: $(u + v) + w = u + (v + w)$;
    \item Identity element: $v + 0_V = v = 0_V + v$;
    \item Commutativity: $u + v = v + u$;
    \item Additive inverse: For every $v$, there exists $-v$ such that $v + (-v) = 0_V$.
  \end{itemize}

  2. Scalar multiplication, often written as $tv$ instead of $t \cdot v$, satisfies:
  \begin{itemize}
    \item Associativity: $s \cdot (t \cdot v) = (st) \cdot v$;
    \item Identity property: $1 \cdot v = v$, where $1$ is the multiplicative identity in $F$.
  \end{itemize}

  3. Scalar multiplication distributes over addition:
  \begin{itemize}
    \item First distributive property: $(s + t) \cdot v = s \cdot v + t \cdot v$;
    \item Second distributive property: $s \cdot (u + v) = s \cdot u + s \cdot v$.
  \end{itemize}

  Where $u, v, w$ (or $s, t$) represent arbitrary elements of $V$ (or $F$). When there is no risk of confusion, we denote $0_V$ simply as $0$, write $u + (-v)$ as $u - v$, and refer to the structure $(V, +, \cdot, 0)$ simply as $V$.
\end{definition}

\begin{definition}
  Let $V$ be an $F$-vector space. If a subset $V_0$ of $V$ contains $0$ and is closed under addition and scalar multiplication, then $(V_0, +, \cdot, 0)$ is also an $F$-vector space, called a subspace of $V$.
\end{definition}