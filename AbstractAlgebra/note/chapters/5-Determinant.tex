\chapter{Determinant}

\section{Permutations Introduction}

\begin{definition}
Let $X$ be a non-empty set. The set of permutations on $X$ is defined as
\[
S_X := \{\, \text{bijections } \sigma : X \to X \,\} .
\]
It contains the identity mapping $\mathrm{id} = \mathrm{id}_X \in S_X$. These permutations can be composed as functions, $(\sigma, \sigma') \mapsto \sigma \sigma'$, or inverted, $\sigma \mapsto \sigma^{-1}$, and the result still belongs to $S_X$.
\end{definition}

\begin{definition}
Fix $n \in \mathbb{Z}_{\geq 1}$. For $1 \leq i \neq j \leq n$, the corresponding transposition $(i\ j) \in S_n$ is defined as the following permutation:
\[
(i\ j):\quad k \mapsto
\begin{cases}
j, & \text{if } k = i, \\
i, & \text{if } k = j, \\
k, & \text{if } k \neq i, j.
\end{cases}
\]
In other words, $(i\ j)$ swaps $i$ and $j$, and leaves all other elements unchanged.
\end{definition}

\begin{definition}
Let $\sigma \in S_n$. The elements of the following set are called the inversions of $\sigma$:
\[
\operatorname{Inv}_\sigma := \{\, (i, j) \in \mathbb{Z}^2 : 1 \leq i < j \leq n,\ \sigma(i) > \sigma(j) \,\}.
\]
The number of inversions of $\sigma$ is defined as $\ell(\sigma) := |\operatorname{Inv}_\sigma|$.
\end{definition}

\begin{proposition}
Let $\sigma \in S_n$. Then there exists $\ell \in \mathbb{Z}_{\geq 0}$ and a sequence of transpositions $\tau_1, \ldots, \tau_\ell \in S_n$ such that
\[
\sigma = \tau_1 \cdots \tau_\ell ;
\]
when $\ell = 0$, the product on the right is understood as the identity $\mathrm{id}$. We call $\ell$ the length of the above decomposition. Among all decompositions of $\sigma$ into transpositions, the minimal possible length is $\ell(\sigma)$.
\end{proposition}

\begin{proposition}
There exists a unique map $\operatorname{sgn} : S_n \to \{\pm 1\}$ such that the following properties hold:
\begin{itemize}
    \item[(i)] For all $\sigma, \xi \in S_n$, we have $\operatorname{sgn}(\sigma\xi) = \operatorname{sgn}(\sigma)\operatorname{sgn}(\xi)$,
    \item[(ii)] If $\tau \in S_n$ is a transposition, then $\operatorname{sgn}(\tau) = -1$.
\end{itemize}
The above map $\operatorname{sgn}$ satisfies $\operatorname{sgn}(\mathrm{id}) = 1$ and $\operatorname{sgn}(\sigma^{-1}) = \operatorname{sgn}(\sigma)^{-1} = \operatorname{sgn}(\sigma)$. Its value can further be expressed in terms of the number of inversions as
\[
\operatorname{sgn}(\sigma) = (-1)^{\ell(\sigma)}.
\]
\end{proposition}

\begin{definition}
Let $\sigma \in S_n$. If there exists a sequence of transpositions $\tau_1, \ldots, \tau_\ell$ such that $\sigma = \tau_1 \cdots \tau_\ell$, where $\ell \in \mathbb{Z}_{\geq 0}$ is even (respectively, odd), then $\sigma$ is called an \emph{even permutation} (respectively, \emph{odd permutation}).
\end{definition}

\begin{definition}
Suppose the concept of orientation has been fixed on $V$. Define the map $D : V^n \to \mathbb{R}$ as follows:
\[
D(u_1, \ldots, u_n) =
\begin{cases}
0, & \text{if } u_1, \ldots, u_n \text{ are dependent}, \\
\text{the volume of } \Diamond(u_1, \ldots, u_n), & \text{if } u_1, \ldots, u_n \text{ are independent and positively }, \\
-\big(\text{the volume of } \Diamond(u_1, \ldots, u_n)\big), & \text{if } u_1, \ldots, u_n \text{ are independent and negatively }.
\end{cases}
\]
We call $D(u_1, \ldots, u_n)$ the oriented volume determined by the ordered set of vectors $u_1, \ldots, u_n \in V$.
\end{definition}

\section{A Characterization of a Class of Alternating Forms}

\begin{definition}
Let $V$ be a vector space over a field $F$. For every $m \in \mathbb{Z}_{\geq 1}$, define
\[
D_{V,m} := \left\{ D : V^m \to F \;\middle|\; \text{$D$ satisfies the following properties D.1 and D.2} \right\}.
\]
\begin{itemize}
    \item[D.1] For each $1 \leq i \leq m$,
    \[
    D(\ldots, v_i + v_i', \ldots) = D(\ldots, v_i, \ldots) + D(\ldots, v_i', \ldots),
    \]
    \[
    D(\ldots, t v_i, \ldots) = t D(\ldots, v_i, \ldots),
    \]
    where $v_i, v_i' \in V$ and $t \in F$, and the ellipsis indicates that the other $m-1$ variables are fixed.
    \item[D.2] If there exist $1 \leq i < j \leq m$ such that $v_i = v_j$, then
    \[
    D(v_1, \ldots, v_m) = 0.
    \]
\end{itemize}
A map in $D_{V,m}$ is also called an $m$-linear alternating form on $V$.
\end{definition}

\begin{lemma}
Let $m \in \mathbb{Z}_{\geq 1}$, $D \in D_{V,m}$, and $\sigma \in S_m$. Then
\[
D\big(v_{\sigma^{-1}(1)}, \ldots, v_{\sigma^{-1}(m)}\big) = \operatorname{sgn}(\sigma) D(v_1, \ldots, v_m),
\]
where $\operatorname{sgn} : S_m \to \{\pm 1\}$ is the sign map.
\end{lemma}

\begin{theorem}
Let $V$ be a finite-dimensional vector space. Then $\dim D_V = 1$. If $n := \dim V \geq 1$ and $e_1, \ldots, e_n$ is an ordered basis of $V$, denoted by $e$, then there exists a unique $D_e \in D_V$ such that $D_e(e_1, \ldots, e_n) = 1$.
\end{theorem}

\section{Definition of Determinant}

\begin{definition}
Let $V$ be an $n$-dimensional vector space, $n \in \mathbb{Z}_{\geq 1}$. For each $T \in \operatorname{End}(V)$, define $\det T \in F$ to be the unique element such that
\[
T^*(D) = (\det T) \cdot D, \quad D \in D_V;
\]
equivalently, for every $D \in D_V$ and $(v_1, \ldots, v_n) \in V^n$,
\[
D(T v_1, \ldots, T v_n) = \det T \cdot D(v_1, \ldots, v_n).
\]
In the case of the zero vector space ($n = 0$), for $T = 0_V = \operatorname{id}_V$, we define $\det(T) := 1$.
\end{definition}

\begin{theorem}
The determinant has the following properties:
\begin{itemize}
    \item[(i)] $\det(\operatorname{id}_V) = 1$.
    \item[(ii)] For $S, T \in \operatorname{End}(V)$, $\det(ST) = \det(S)\det(T)$.
    \item[(iii)] If $T$ is invertible, then $\det T \in F$ is also invertible, and $\det(T^{-1}) = (\det T)^{-1}$.
\end{itemize}
\end{theorem}

\begin{proposition}
Let $T \in \operatorname{End}(V)$ and let $S : V \xrightarrow{\sim} W$ be an isomorphism of finite-dimensional vector spaces. Then $STS^{-1} \in \operatorname{End}(W)$ satisfies
\[
\det(STS^{-1}) = \det T.
\]
\end{proposition}

\begin{proposition}
Let $V$ be an $n$-dimensional vector space (with $n \geq 1$), and let $e_1, \ldots, e_n$ be an ordered basis of $V$, denoted by $e$. Let $D_e \in D_V$ be as given in Theorem 5.3.5. Then
\[
\det T = D_e(T e_1, \ldots, T e_n).
\]
\end{proposition}

\begin{proposition}
Let $e$ be the standard ordered basis of $F^n$, $n \in \mathbb{Z}_{\geq 1}$. In this way, each $A = (a_{ij})_{i,j} \in M_{n \times n}(F)$ is identified with an element of $\operatorname{End}(F^n)$. The determinant of the matrix $A$ is defined as
\[
\det A := D_e(Ae_1, \ldots, Ae_n),
\]
Then,
\[
\det A = \sum_{\sigma \in S_n} \operatorname{sgn}(\sigma) a_{\sigma(1),1} \cdots a_{\sigma(n),n}
= \sum_{\sigma \in S_n} \operatorname{sgn}(\sigma) a_{1,\sigma(1)} \cdots a_{n,\sigma(n)}.
\]
\end{proposition}

\begin{definition}
Let $1 \leq i, j \leq n$, where $n \in \mathbb{Z}_{\geq 1}$. The $(i, j)$-th minor of a matrix $A \in M_{n \times n}(F)$ is defined as the determinant of the $(n-1) \times (n-1)$ matrix $M_{ij}$ obtained by deleting the $i$-th row and $j$-th column from $A$. It is denoted by
\[
M_{ij} := \det M_{ij} \in F.
\]
\end{definition}

\section{Cramer's Rule}

\begin{proposition}
Let $V$ be a finite-dimensional $F$-vector space. Then $T \in \operatorname{End}(V)$ is invertible if and only if $\det T \in F^\times$.
\end{proposition}

\begin{corollary}
Let $v_1, \ldots, v_n$ be elements of an $n$-dimensional $F$-vector space $V$ ($n \in \mathbb{Z}_{\geq 1}$). Then the following statements are equivalent:
\begin{itemize}
    \item[(i)] $v_1, \ldots, v_n$ are linearly dependent;
    \item[(ii)] $D(v_1, \ldots, v_n) = 0$ for all $D \in D_V$;
    \item[(iii)] There exists an ordered basis $e_1, \ldots, e_n$ of $V$, denoted by $e$, such that $D_e(v_1, \ldots, v_n) = 0$.
\end{itemize}
\end{corollary}

\begin{definition}
Let $n \in \mathbb{Z}_{\geq 1}$. For $A = (a_{ij})_{i,j} \in M_{n \times n}(F)$, the classical adjugate matrix is defined as
\[
A^\vee = (A_{ji})_{i,j} \in M_{n \times n}(F),
\]
where
\[
A_{ij} := (-1)^{i+j} M_{ij}, \quad 1 \leq i, j \leq n,
\]
and $M_{ij}$ is the $(i,j)$-th minor of $A$.
\end{definition}

\begin{theorem}
For any $A \in M_{n \times n}(F)$, we have
\[
A A^\vee = \det A \cdot 1_{n \times n} = A^\vee A.
\]
\end{theorem}

\begin{corollary}[Cramer's Rule]
Consider the system of $n$ linear equations over a field $F$:
\[
\begin{cases}
a_{11} X_1 + \cdots + a_{1n} X_n = b_1 \\
\quad\vdots \\
a_{n1} X_1 + \cdots + a_{nn} X_n = b_n
\end{cases}
\]
Let the coefficient matrix $A = (a_{ij})_{i,j} \in M_{n \times n}(F)$ be regarded as a linear map from $F^n$ to $F^n$.
\begin{itemize}
    \item[(i)] If the system is homogeneous, i.e., $b_1 = \cdots = b_n = 0$, then its solution set is $\ker A$ (see Definition 4.8.1). In particular, the system has a nontrivial solution if and only if $\det A = 0$.
    \item[(ii)] When $(b_1, \ldots, b_n) \in F^n$ is given, the system either has no solution, or its solution set is of the form
    \[
    (x_1, \ldots, x_n) + \ker A,
    \]
    where $(x_1, \ldots, x_n) \in F^n$ is any particular solution.
    \item[(iii)] If $\det A \in F^\times$, then for any $(b_1, \ldots, b_n) \in F^n$, the system has a unique solution $(x_1, \ldots, x_n)$, where
    \[
    x_j = \frac{
    \begin{vmatrix}
    a_{11} & \cdots & b_1 & \cdots & a_{1n} \\
    a_{21} & \cdots & b_2 & \cdots & a_{2n} \\
    \vdots &        & \vdots &        & \vdots \\
    a_{n1} & \cdots & b_n & \cdots & a_{nn}
    \end{vmatrix}
    }{
    \begin{vmatrix}
    a_{11} & \cdots & a_{1j} & \cdots & a_{1n} \\
    a_{21} & \cdots & a_{2j} & \cdots & a_{2n} \\
    \vdots &        & \vdots &        & \vdots \\
    a_{n1} & \cdots & a_{nj} & \cdots & a_{nn}
    \end{vmatrix}
    }, \quad j = 1, \ldots, n,
    \]
    where in the numerator, the $j$-th column of $A$ is replaced by the column vector $(b_1, \ldots, b_n)^T$.
\end{itemize}
\end{corollary}

\begin{note}
Although Cramer's rule provides an exact solution to a system of linear equations when the coefficient matrix is invertible, in practice the computational cost of evaluating determinants increases rapidly with $n$. Even when these determinants can be computed efficiently, Cramer's rule remains numerically unstable: since it involves division, when the denominator $\det A$ is close to zero, even a \textbf{small perturbation can cause large changes} in $x_1, \ldots, x_n$. Therefore, the value of Cramer's rule lies mainly in its theoretical significance.
\end{note}
