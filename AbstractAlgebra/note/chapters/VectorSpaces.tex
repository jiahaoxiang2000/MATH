\chapter{Vector Spaces and linear mappings}
\label{chap:VectorSpaces}

Broadly speaking, a vector space over a field $F$ refers to a set $V$ together with two operations:
\begin{itemize}
  \item Vector addition $+: V \times V \rightarrow V$, denoted $(v_1, v_2) \mapsto v_1 + v_2$, satisfying associativity, commutativity, and the existence of inverses;
  \item Scalar multiplication $\cdot: F \times V \rightarrow V$, denoted $(t, v) \mapsto t \cdot v = tv$, satisfying associativity and distributivity over addition.
\end{itemize}

If a mapping $T: V \rightarrow W$ between vector spaces satisfies the identities
\begin{align*}
  T(v_1 + v_2) &= T(v_1) + T(v_2),\\
  T(tv) &= tT(v),
\end{align*}
then $T$ is called a linear mapping.

\section{Introduction: Back to the system of linear equations}

\begin{align}
  a_{11}X_1 + \cdots + a_{1n}X_n &= b_1 \nonumber\\
  a_{21}X_1 + \cdots + a_{2n}X_n &= b_2 \nonumber \\
  &\vdots\nonumber \\
  a_{m1}X_1 + \cdots + a_{mn}X_n &= b_m \label{eq:linear-system-2}
\end{align}

\begin{definition}
  Consider a system of $n$ linear equations over a field $F$ in the form (\ref{eq:linear-system-2}). If $b_1 = \cdots = b_m = 0$, then the system is called homogeneous.
\end{definition}

Given $n,m \in \mathbb{Z}_{\geq 1}$ and a family of coefficients $(a_{ij})_{1 \leq i \leq m, 1 \leq j \leq n}$ where $a_{ij} \in F$, define the mapping
\begin{align*}
  T: F^n &\rightarrow F^m\\
  (x_j)_{j=1}^n &\mapsto \left(\sum_{j=1}^n a_{1j}x_j, \ldots, \sum_{j=1}^n a_{mj}x_j\right).
\end{align*}

\begin{definition}
  Let $T: F^n \rightarrow F^m$ correspond to a homogeneous system of linear equations as described above. If $v_1, \ldots, v_h \in F^n$ are all solutions of the system, and every solution $x \in F^n$ can be uniquely expressed through addition and scalar multiplication as
  \begin{align}
    x = \sum_{i=1}^h t_i v_i, \quad t_1, \ldots, t_h \in F,
  \end{align}
  where the tuple $(t_1, \ldots, t_h)$ is uniquely determined by $x$, then $v_1, \ldots, v_h$ is called a fundamental system of solutions for the homogeneous system.
\end{definition}

\begin{proposition}
  Consider a homogeneous system of $n$ linear equations in the form (Eq.\ref{eq:linear-system-2}), where $\bf{b} = 0$. If the reduced row echelon matrix obtained by elimination has $r$ pivot elements, then the corresponding homogeneous system has a fundamental system of solutions $v_1, \ldots, v_{n-r}$.
\end{proposition}

\section{Vector spaces}

\begin{definition}
  A vector space over a field $F$, also called an $F$-vector space, is a tuple $(V, +, \cdot, 0_V)$ where $V$ is a set, $0_V \in V$, and operations $+: V \times V \rightarrow V$ and $\cdot: F \times V \rightarrow V$ are written as $(u,v) \mapsto u + v$ and $(t,v) \mapsto t \cdot v$ respectively, satisfying the following conditions:

  1. Addition satisfies:
  \begin{itemize}
    \item Associativity: $(u + v) + w = u + (v + w)$;
    \item Identity element: $v + 0_V = v = 0_V + v$;
    \item Commutativity: $u + v = v + u$;
    \item Additive inverse: For every $v$, there exists $-v$ such that $v + (-v) = 0_V$.
  \end{itemize}

  2. Scalar multiplication, often written as $tv$ instead of $t \cdot v$, satisfies:
  \begin{itemize}
    \item Associativity: $s \cdot (t \cdot v) = (st) \cdot v$;
    \item Identity property: $1 \cdot v = v$, where $1$ is the multiplicative identity in $F$.
  \end{itemize}

  3. Scalar multiplication distributes over addition:
  \begin{itemize}
    \item First distributive property: $(s + t) \cdot v = s \cdot v + t \cdot v$;
    \item Second distributive property: $s \cdot (u + v) = s \cdot u + s \cdot v$.
  \end{itemize}

  Where $u, v, w$ (or $s, t$) represent arbitrary elements of $V$ (or $F$). When there is no risk of confusion, we denote $0_V$ simply as $0$, write $u + (-v)$ as $u - v$, and refer to the structure $(V, +, \cdot, 0)$ simply as $V$.
\end{definition}

\begin{definition}
  Let $V$ be an $F$-vector space. If a subset $V_0$ of $V$ contains $0$ and is closed under addition and scalar multiplication, then $(V_0, +, \cdot, 0)$ is also an $F$-vector space, called a subspace of $V$.
\end{definition}

\section{Matrix \& calculate}

\begin{definition}
  Let $m,n \in \mathbb{Z}_{\geq 1}$. An $m \times n$ matrix over a field $F$ is a rectangular array
  \begin{align*}
    A = (a_{ij})_{1 \leq i \leq m, 1 \leq j \leq n} =
    \begin{pmatrix}
      a_{11} & \cdots & a_{1n} \\
      \vdots & \ddots & \vdots \\
      a_{m1} & \cdots & a_{mn}
    \end{pmatrix} =
    \begin{bmatrix}
      \cdots & \cdots & a_{ij} & \cdots & \cdots \\
    \end{bmatrix}
  \end{align*}

  \noindent where $a_{ij} \in F$ is called the $(i,j)$-entry or $(i,j)$-element of matrix $A$, with $i$ indicating the row and $j$ indicating the column. An $n \times n$ matrix is called a square matrix of order $n$.

  We denote the set of all $m \times n$ matrices over $F$ by $M_{m \times n}(F)$.
\end{definition}

\begin{proposition}
  The set $M_{m \times n}(F)$ equipped with standard addition and scalar multiplication forms an $F$-vector space. The zero element is the zero matrix, and the additive inverse of a matrix $A = (a_{ij})_{i,j}$ is $-A = (-a_{ij})_{i,j}$.
\end{proposition}

\begin{definition}[Matrix Multiplication]
  Matrix multiplication is a mapping defined as:
  \begin{align*}
    M_{m \times n}(F) \times M_{n \times r}(F) &\rightarrow M_{m \times r}(F)\\
    (A, B) &\mapsto AB
  \end{align*}

  If $A = (a_{ij})_{1 \leq i \leq m, 1 \leq j \leq n}$ and $B = (b_{jk})_{1 \leq j \leq n, 1 \leq k \leq r}$, then $AB = (c_{ik})_{1 \leq i \leq m, 1 \leq k \leq r}$, where:
  \begin{align*}
    c_{ik} := \sum_{j=1}^{n} a_{ij}b_{jk} =
    \begin{pmatrix} a_{i1} & \cdots & a_{in}
    \end{pmatrix}
    \begin{pmatrix} b_{1k} \\ \vdots \\ b_{nk}
    \end{pmatrix}
  \end{align*}

  This represents the dot product of the $i^{th}$ row of $A$ with the $k^{th}$ column of $B$.
\end{definition}

\begin{proposition}
  Matrix multiplication satisfies the following properties:
  \begin{itemize}
    \item Associativity: $(AB)C = A(BC)$;
    \item Distributivity: $A(B + C) = AB + AC$ and $(B + C)A = BA + CA$;
    \item Linearity: $A(tB) = t(AB) = (tA)B$;
  \end{itemize}
  where $t \in F$ and matrices $A, B, C$ are arbitrary, provided their dimensions make these operations valid.
\end{proposition}

\section{Bases \& Dimensions}

\begin{definition}
  Let $S$ be a subset of an $F$-vector space $V$.
  \begin{itemize}
    \item If $\langle S \rangle = V$, then $S$ is said to generate $V$, or $S$ is called a generating set of $V$.
    \item A linear relation in $S$ is an equation of the form
      \begin{align*}
        \sum_{s \in S} a_s s = 0
      \end{align*}
      This relation is called trivial if all coefficients $a_s$ are zero; otherwise, it is non-trivial. The set $S$ is linearly dependent if there exists a non-trivial linear relation among its elements; otherwise, $S$ is linearly independent.
    \item If $S$ is a linearly independent generating set, then $S$ is called a basis of $V$.
  \end{itemize}
\end{definition}

\begin{lemma}
  For any subset $S$ of an $F$-vector space $V$, the following statements are equivalent:
  \begin{enumerate}
    \item $S$ is a minimal generating set.
    \item $S$ is a basis.
    \item $S$ is a maximal linearly independent subset.
  \end{enumerate}
\end{lemma}

\begin{proof}
  Let's prove the equivalence by showing $(1) \Rightarrow (2)$, $(2) \Rightarrow (3)$, and $(3) \Rightarrow (1)$.

  $(1) \Rightarrow (2)$: If $S$ is a minimal generating set, then $\langle S \rangle = V$. Suppose $S$ is not linearly independent. Then there exists some $s_0 \in S$ that can be expressed as a linear combination of other elements in $S$. But this means $S \setminus \{s_0\}$ still generates $V$, contradicting the minimality of $S$. Therefore, $S$ must be linearly independent, making it a basis.

  $(2) \Rightarrow (3)$: Let $S$ be a basis. Then $S$ is linearly independent and $\langle S \rangle = V$. To show that $S$ is maximal, suppose we add any vector $v \not\in S$ to form $S' = S \cup \{v\}$. Since $\langle S \rangle = V$, we have $v \in \langle S \rangle$, meaning $v$ can be written as a linear combination of elements in $S$. Therefore, $S'$ must be linearly dependent, proving that $S$ is a maximal linearly independent set.

  $(3) \Rightarrow (1)$: Let $S$ be a maximal linearly independent set. If $\langle S \rangle \neq V$, then there exists some $v \in V \setminus \langle S \rangle$. The set $S \cup \{v\}$ would still be linearly independent, contradicting the maximality of $S$. Therefore $\langle S \rangle = V$. Now suppose $S$ is not minimal. Then there exists a proper subset $S' \subset S$ with $\langle S' \rangle = V$. But this means some element in $S \setminus S'$ can be expressed as a linear combination of elements in $S'$, making $S$ linearly dependent, which is a contradiction. Thus, $S$ is a minimal generating set.
\end{proof}