\chapter{Determinant}

\section{Permutations Introduction}

\begin{definition}
Let $X$ be a non-empty set. The set of permutations on $X$ is defined as
\[
S_X := \{\, \text{bijections } \sigma : X \to X \,\} .
\]
It contains the identity mapping $\mathrm{id} = \mathrm{id}_X \in S_X$. These permutations can be composed as functions, $(\sigma, \sigma') \mapsto \sigma \sigma'$, or inverted, $\sigma \mapsto \sigma^{-1}$, and the result still belongs to $S_X$.
\end{definition}

\begin{definition}
Fix $n \in \mathbb{Z}_{\geq 1}$. For $1 \leq i \neq j \leq n$, the corresponding transposition $(i\ j) \in S_n$ is defined as the following permutation:
\[
(i\ j):\quad k \mapsto
\begin{cases}
j, & \text{if } k = i, \\
i, & \text{if } k = j, \\
k, & \text{if } k \neq i, j.
\end{cases}
\]
In other words, $(i\ j)$ swaps $i$ and $j$, and leaves all other elements unchanged.
\end{definition}

\begin{definition}
Let $\sigma \in S_n$. The elements of the following set are called the inversions of $\sigma$:
\[
\operatorname{Inv}_\sigma := \{\, (i, j) \in \mathbb{Z}^2 : 1 \leq i < j \leq n,\ \sigma(i) > \sigma(j) \,\}.
\]
The number of inversions of $\sigma$ is defined as $\ell(\sigma) := |\operatorname{Inv}_\sigma|$.
\end{definition}

\begin{proposition}
Let $\sigma \in S_n$. Then there exists $\ell \in \mathbb{Z}_{\geq 0}$ and a sequence of transpositions $\tau_1, \ldots, \tau_\ell \in S_n$ such that
\[
\sigma = \tau_1 \cdots \tau_\ell ;
\]
when $\ell = 0$, the product on the right is understood as the identity $\mathrm{id}$. We call $\ell$ the length of the above decomposition. Among all decompositions of $\sigma$ into transpositions, the minimal possible length is $\ell(\sigma)$.
\end{proposition}

\begin{proposition}
There exists a unique map $\operatorname{sgn} : S_n \to \{\pm 1\}$ such that the following properties hold:
\begin{itemize}
    \item[(i)] For all $\sigma, \xi \in S_n$, we have $\operatorname{sgn}(\sigma\xi) = \operatorname{sgn}(\sigma)\operatorname{sgn}(\xi)$,
    \item[(ii)] If $\tau \in S_n$ is a transposition, then $\operatorname{sgn}(\tau) = -1$.
\end{itemize}
The above map $\operatorname{sgn}$ satisfies $\operatorname{sgn}(\mathrm{id}) = 1$ and $\operatorname{sgn}(\sigma^{-1}) = \operatorname{sgn}(\sigma)^{-1} = \operatorname{sgn}(\sigma)$. Its value can further be expressed in terms of the number of inversions as
\[
\operatorname{sgn}(\sigma) = (-1)^{\ell(\sigma)}.
\]
\end{proposition}

\begin{definition}
Let $\sigma \in S_n$. If there exists a sequence of transpositions $\tau_1, \ldots, \tau_\ell$ such that $\sigma = \tau_1 \cdots \tau_\ell$, where $\ell \in \mathbb{Z}_{\geq 0}$ is even (respectively, odd), then $\sigma$ is called an \emph{even permutation} (respectively, \emph{odd permutation}).
\end{definition}
