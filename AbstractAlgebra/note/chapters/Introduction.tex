\chapter{Introduction}

\section{What is Algebra?}

In light of this, classical algebra can be understood as the art of solving equations by:
\begin{itemize}
  \item Replacing specific numbers with variables
  \item Using operations such as transposition of terms
\end{itemize}

This traditional approach forms the foundation of algebraic manipulation and equation solving.

\begin{theorem}[Fundamental Theorem of Algebra]
  Let $f = X^n + a_{n-1}X^{n-1} + \cdots + a_0$ be a polynomial in $X$ with complex coefficients, where $n \in \mathbb{Z}_{\geq 1}$. Then there exist $x_1,\ldots,x_n \in \mathbb{C}$ such that:
  \[
    f = \prod_{k=1}^n (X-x_k)
  \]
  These $x_1,\ldots,x_n$ are precisely the complex roots of $f$ (counting multiplicity); they are unique up to reordering.
\end{theorem}

Now let us further explain the previously raised question: What is algebra?

\begin{itemize}
  \item \textbf{What is an equation?} \\
    An expression obtained through a finite number of basic operations: addition, subtraction, multiplication, and division (with non-zero denominators).

  \item \textbf{What are numbers?} \\
    At minimum, this includes common number systems like $\mathbb{Q}$, $\mathbb{R}$, and $\mathbb{C}$. All these systems support four basic operations, though division requires non-zero denominators. Note that $\mathbb{Z}$ is not included in this list, as division is not freely applicable in $\mathbb{Z}$.

  \item \textbf{What is the art of solving?} \\
    This involves:
    \begin{itemize}
      \item Determining whether equations have solutions
      \item Finding exact solutions when possible
      \item Developing efficient algorithms, Providing methods for approximating solutions
    \end{itemize}
\end{itemize}